\documentclass[]{article}
\usepackage{multicol}
\usepackage{setspace}

%opening
\title{Signals and Systems - Formulae and Identities}
\begin{document}
\maketitle

\begin{multicols}{2}
\section{Basics}
\subsection{}
The normalized energy content $E$ of a signal $x(t)$ is defined as

\[ E = \int_{-\infty}^{\infty} |x(t)|^{2}dt \]

\subsection{}
The normalized average power $P$ of a signal $x(t)$ is defined as

\[ P = \lim_{T \to \infty} \frac{1}{T} \int_{-T/2}^{T/2} |x(t)|^{2} dt \]

\subsection{}
The unit impulse function (or Dirac delta function) $\delta (t)$ is defined as

\[\delta (t) = \int_{-\infty}^{\infty} \phi(t) \delta (t)dt = \phi(0)\]
where $\phi (t) $ is any test function continuous at $t=0$. The unit impulse function is a \textit{generalized function}.

\subsection{}
The derivative $g'(t)$ of a generalized function $g(t)$ is defined by
\[\int_{-\infty}^{\infty} g'(t) \phi(t) dt = - \int_{-\infty}^{\infty} g(t) \phi '(t)dt\]

\subsection{}
The Fourier series for a signal $x(t)$ is defined as

\[ x(t) = \sum_{n=-\infty}^{\infty} c_{n} e^{jn\omega_{0} t} \]
where $\omega_{0}$ is the fundamental angular frequency.
The Fourier coefficients $c_{n}$ are given by

\[ c_{n} = \frac{1}{T_{0}} \int_{-T_{0}/2}^{T_{0}/2} x(t) e^{-jn\omega_{0}t} dt \]
A plot of $|c_{n}|$ vs $\omega$ is called the amplitude spectrum. A plot of $\theta_{n}$ (the phase constants of $c_{n}$) vs $\omega$ is called the phase spectrum. Together these are referred to as the frequency spectra.
\subsection{}
Parseval's theorem states that for a periodic signal $x(t)$
\[ \frac{1}{T_{0}} \int_{-T_{0}/2}^{T_{0}/2} |x(t)|^2dt = \sum_{n=-\infty}^{\infty} |c_{n}|^2 \]

\subsection{}
The Fourier transform, $\mathcal{F}$, of a signal $x(t)$ is given by
\[ X(\omega) = \mathcal{F}[x(t)] = \int_{-\infty}^{\infty}x(t)e^{-j\omega t}dt \]

\subsection{}
The inverse Fourier transform of $X(\omega)$,  $\mathcal{F}^{-1}$, is given by
\[ x(t) = \frac{1}{2\pi} \int_{-\infty}^{\infty}X(\omega)e^{j\omega t} d\omega \]

\section{Properties of the Fourier Transform}
$ x(t) \longleftrightarrow X(\omega)$ denotes a Fourier transform pair.
\subsection{Linearity}
\[ a_{1} x_{1}(t) + a_{2} x_{2}(t) \longleftrightarrow a_{1} X_{1}(\omega) + a_{2} X_{2}(\omega) \]
\subsection{Time Shifting}
\[ x(t-t_{0}) \longleftrightarrow X(\omega)e^{-j\omega t_{0}} \]
\subsection{Frequency Shifting}
\[ x(t)e^{j\omega_{0} t} \longleftrightarrow X(\omega - \omega_{0}) \]
\subsection{Scaling}
\[ x(at) \longleftrightarrow \frac{1}{|a|}X(\frac{\omega}{a}) \]
\subsection{Time Reversal}
\[ x(-t) \longleftrightarrow X(-\omega) \]
\subsection{Duality}
\[ X(t) \longleftrightarrow 2\pi x(-\omega) \]
\subsection{Differentiation}
Time differentiation
\[ x'(t) = \frac{d}{dt}x(t) \longleftrightarrow j\omega X(\omega) \]
Frequency differentiation
\[ (-jt)x(t) \longleftrightarrow X'(\omega) = \frac{d}{d\omega}X(\omega) \]
\subsection{Integration}
\[ \int_{-\infty}^{t} x(\tau)d\tau \longleftrightarrow \frac{1}{j\omega}X(\omega) + \pi X(0)\delta(\omega) \]

\section{Convolutions}


\end{multicols}
\end{document}
