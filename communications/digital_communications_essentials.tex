\documentclass[]{article}
\usepackage{multicol}
\usepackage{setspace}
\usepackage{mathrsfs}
\usepackage{graphicx}
\usepackage{subcaption}
\usepackage[labelfont=bf]{caption}

\setcounter{tocdepth}{2} 

%opening
\title{Analog and Digital Communications - Formulae and Identities}
\begin{document}
\maketitle

\tableofcontents
\newpage

\section{Signals and Systems}
%\begin{multicols}{2}
	
\subsection{Basics}
\subsubsection{Normalized Energy}
The normalized energy content $E$ of a signal $x(t)$ is defined as

\begin{equation} E = \int_{-\infty}^{\infty} |x(t)|^{2}dt \label{norm_energy} \end{equation}

\subsubsection{Average Power}
The normalized average power $P$ of a signal $x(t)$ is defined as

\begin{equation} P = \lim_{T \to \infty} \frac{1}{T} \int_{-T/2}^{T/2} |x(t)|^{2} dt\label{norm_power}\end{equation} 

\subsubsection{Dirac Delta Function}
The unit impulse function (or Dirac delta function) $\delta (t)$ is defined as

\begin{equation}\delta (t) = \int_{-\infty}^{\infty} \phi(t) \delta (t)dt = \phi(0)\label{dirac_def}\end{equation}
where $\phi (t) $ is any test function continuous at $t=0$. The unit impulse function is a \textit{generalized function}.

\subsubsection{Derivatives of Generalized Functions}
The derivative $g'(t)$ of a generalized function $g(t)$ is defined by
\begin{equation}\int_{-\infty}^{\infty} g'(t) \phi(t) dt = - \int_{-\infty}^{\infty} g(t) \phi '(t)dt\label{gen_func_def} \end{equation}

\subsubsection{Complex Fourier Series}
The Fourier series for a signal $x(t)$ is defined as

\begin{equation} x(t) = \sum_{n=-\infty}^{\infty} c_{n} e^{jn\omega_{0}\label{complex_fourier_series} t}\end{equation}
where $\omega_{0}$ is the fundamental angular frequency.
The Fourier coefficients $c_{n}$ are given by

\begin{equation} c_{n} = \frac{1}{T_{0}} \int_{-T_{0}/2}^{T_{0}/2} x(t) e^{-jn\omega_{0}t} dt\label{complex_fourier_coefs} \end{equation}
A plot of $|c_{n}|$ vs $\omega$ is called the amplitude spectrum. A plot of $\theta_{n}$ (the phase constants of $c_{n}$) vs $\omega$ is called the phase spectrum. Together these are referred to as the frequency spectra.
\subsubsection{Parseval's Theorem}
Parseval's theorem states that for a periodic signal $x(t)$
\begin{equation} \frac{1}{T_{0}} \int_{-T_{0}/2}^{T_{0}/2} |x(t)|^2dt = \sum_{n=-\infty}^{\infty} |c_{n}|^2\label{parsevals_theorem} \end{equation}

\subsubsection{Fourier Transform}
The Fourier transform, $\mathscr{F}$, of a signal $x(t)$ is given by
\begin{equation} X(\omega) = \mathscr{F}[x(t)] = \int_{-\infty}^{\infty}x(t)e^{-j\omega t}dt\label{complex_fourier_transform} \end{equation}

\subsubsection{Inverse Fourier Transform}
The inverse Fourier transform of $X(\omega)$,  $\mathscr{F}^{-1}$, is given by
\begin{equation} x(t) = \frac{1}{2\pi} \int_{-\infty}^{\infty}X(\omega)e^{j\omega t} d\omega\label{inv_complex_fourier_transform} \end{equation}

\subsection{Properties of the Fourier Transform}
$ x(t) \longleftrightarrow X(\omega)$ denotes a Fourier transform pair.
\subsubsection{Linearity}
\begin{equation} a_{1} x_{1}(t) + a_{2} x_{2}(t) \longleftrightarrow a_{1} X_{1}(\omega) + a_{2} X_{2}(\omega)\label{fourier_linearity} \end{equation}
\subsubsection{Time Shifting}
\begin{equation} x(t-t_{0}) \longleftrightarrow X(\omega)e^{-j\omega t_{0}}\label{fourier_time_shifting} \end{equation}
\subsubsection{Frequency Shifting}
\begin{equation} x(t)e^{j\omega_{0} t} \longleftrightarrow X(\omega - \omega_{0})\label{fourier_frequency_shifting} \end{equation}
\subsubsection{Scaling}
\begin{equation} x(at) \longleftrightarrow \frac{1}{|a|}X(\frac{\omega}{a})\label{fourier_scaling} \end{equation}
\subsubsection{Time Reversal}
\begin{equation} x(-t) \longleftrightarrow X(-\omega)\label{fourier_time_reversal} \end{equation}
\subsubsection{Duality}
\begin{equation} X(t) \longleftrightarrow 2\pi x(-\omega)\label{fourier_duality} \end{equation}
\subsubsection{Differentiation}
Time differentiation
\begin{equation} x'(t) = \frac{d}{dt}x(t) \longleftrightarrow j\omega X(\omega)\label{fourier_time_differentiation}\end{equation}
Frequency differentiation
\begin{equation} (-jt)x(t) \longleftrightarrow X'(\omega) = \frac{d}{d\omega}X(\omega)\label{fourier_freq_differentiation} \end{equation}
\subsubsection{Integration}
\begin{equation} \int_{-\infty}^{t} x(\tau)d\tau \longleftrightarrow \frac{1}{j\omega}X(\omega) + \pi X(0)\delta(\omega)\label{fourier_integration}\end{equation}

\subsubsection{Modulation Theorem}
\begin{equation} x(t)cos(\omega_{0}t) \longleftrightarrow \frac{1}{2}X(\omega - \omega_{0}) + \frac{1}{2}X(\omega + \omega_{0})\label{modulation_theorem}\end{equation}

\subsection{Convolutions and Correlation}
The convolution of two signals $x_{1}(t)$ and $x_{2}(t)$ is
\begin{equation} x_{1}(t) * x_{2}(t) = \int_{-\infty}^{\infty} x_{1}(\tau) x_{2}(t-\tau)d\tau\label{convolution_def} \end{equation}
\subsubsection{Time Convolution Theorem}
\begin{equation} x_{1}(t) * x_{2}(t) \longleftrightarrow X_{1}(\omega) X_{2}(\omega)\label{time_convolution_thrm}\end{equation} 
\subsubsection{Frequency Convolution Theorem}
\begin{equation}x_{1}(t)x_{2}(t) \longleftrightarrow \frac{1}{2\pi}X_{1}(\omega)*X_{2}(\omega)\label{freq_convolution_thrm} \end{equation}
\subsubsection{Cross-Correlation}
The cross correlation $R_{12}(\tau)$ of signals $x_{1}(t)$ and $x_{2}(t)$ is defined by
\begin{equation}R_{12}(\tau) = \int_{-\infty}^{\infty}x_{1}(t)x_{2}(t-\tau)dt\label{cross_corr_defn} \end{equation}
\subsubsection{Autocorrelation}
The autocorrelation is defined as the cross-correlation of a signal $x_{1}(t)$ with itself, $R_{11}(\tau)$.
\subsubsection{Energy Spectral Density}
The energy spectral density $S_{11}$ of a signal $x_{1}(t)$ is given by
\begin{equation}S_{11}(\omega) = \mathscr{F}[R_{11}(\tau)] = \int_{-\infty}^{\infty}R_{11}(\tau)e^{-j\omega \tau}d\omega\label{energy_spectral_density} \end{equation}
\subsection{Linear Time-Invariant Systems}
Linear time-invariant (linear time-invariant) systems have several properties, as follows. Suppose $\mathcal{F}$ is an operator representing the action of a system with output $y(t)$.
\subsubsection{Additivity}
\begin{equation}\mathcal{F}[x_{1}(t) +x_{2}(t)] = \mathcal{F}[x_{1}(t)] + \mathcal{F}[x_{2}(t)]\label{lti_additivity}\end{equation}
\subsubsection{Homogeneity}
\begin{equation}\mathcal{F}[ax(t)] = a\mathcal{F}[x(t)]\label{lti_homogeneity} \end{equation}
\subsubsection{Time-Invariance}
\begin{equation}\mathcal{F}[x(t-t_{0})] = y(t-t_{0})\label{lti_time_invariance} \end{equation}
\subsubsection{Impulse Response}
The impulse response $h(t)$ of an LTI system is the response of the system with a delta function input
\begin{equation} h(t) = \mathcal{F}[\delta(t)] \label{impulse_response_def}\end{equation}
\subsubsection{Response to Arbitrary Inputs}
The response of an LTI system to an arbitrary input can be expressed in terms of a convolution with the impulse response of the system
\begin{equation}y(t) = x(t)*h(t)=\int_{-\infty}^{\infty}x(\tau)h(t-\tau)d\tau\label{lti_system_response} \end{equation}
\subsubsection{Causality}
A signal $x(t)$ is causal if, for $t<0$, $x(t)=0$.
\subsubsection{Frequency Response}
Using the time convolution theorem ($\ref{time_convolution_thrm}$) on the response of an LTI system ($\ref{lti_system_response}$), we find that
\begin{equation} Y(\omega) = X(\omega)H(\omega) \label{freq_response} \end{equation}
where $Y(\omega)=\mathscr{F}[y(t)]$ and $H(\omega)=\mathscr{F}[h(t)]$. We refer to $H(\omega)$ as the \textit{frequency response} or \textit{transfer function}.
\subsubsection{Input and Output Spectral Densities}
\begin{equation} S_{yy}(\omega) = |H(\omega)|^{2} S_{xx}(\omega)  \end{equation}
\begin{equation}\bar{S}_{yy}(\omega) = |H(\omega)|^{2} \bar{S}_{xx}(\omega)  \end{equation}

%\end{multicols}
\newpage
\section{Amplitude Modulation}
Modulation is defined as the process by which some characteristic of a carrier signal is varied in accordance with a modulating signal (message signal). There are two basic types of analog modulation; continuous wave (CW) modulation and pulse modulation.

%\begin{multicols}{2}
\subsection{Continuous-Wave Modulation}
In continuous-wave modulation, a sinusoidal signal is used as a carrier signal. The modulated carrier signal $x_{c}(t)$ is given by
\begin{equation} x_{c}(t) = A(t)cos[\omega_{c}t + \phi (t)] \label{cw_carrier_signal} \end{equation}
Let $m(t)$ be the message signal.
\subsubsection{Double-Sideband Modulation}
Double-sideband (DSB) modulation occurs when $m(t) = A(t)$. The DSB modulated signal $x_{DSB}(t)$ is then given by
\begin{equation} x_{DSB}(t) = m(t)cos(\omega_{c}t) \label{dsb_carrier_signal} \end{equation}
By applying the modulation theorem ($\ref{modulation_theorem}$), we obtain the following
\begin{equation} X_{DSB}(\omega) = \frac{1}{2}M(\omega - \omega_{c}) + \frac{1}{2}M(\omega + \omega_{c}) \label{dsb_fourier_transform} \end{equation}
See figure $\ref{fig:dsb_modulation}$ for an example.

\begin{figure}[h!]
	\centering
	\begin{subfigure}[b]{0.3\textwidth}
		\includegraphics[width=\textwidth]{figs/amplitude_modulation/dsb/message_signal.png}
		\caption{Message signal}
		\label{fig:dsb_message_signal}
	\end{subfigure}
	~ %add desired spacing between images, e. g. ~, \quad, \qquad, \hfill etc. 
	%(or a blank line to force the subfigure onto a new line)
	\begin{subfigure}[b]{0.3\textwidth}
		\includegraphics[width=\textwidth]{figs/amplitude_modulation/dsb/carrier_signal.png}
		\caption{Carrier signal}
		\label{fig:dsb_carrier_signal}
	\end{subfigure}
	~ %add desired spacing between images, e. g. ~, \quad, \qquad, \hfill etc. 
	%(or a blank line to force the subfigure onto a new line)
	\begin{subfigure}[b]{0.3\textwidth}
		\includegraphics[width=\textwidth]{figs/amplitude_modulation/dsb/modulated_signal.png}
		\caption{Modulated signal}
		\label{fig:dsb_modulated_signal}
	\end{subfigure}
	\begin{subfigure}[b]{0.3\textwidth}
		\includegraphics[width=\textwidth]{figs/amplitude_modulation/dsb/message_signal_freq.png}
		\caption{Message signal}
		\label{fig:dsb_message_signal_freq}
	\end{subfigure}
	~ %add desired spacing between images, e. g. ~, \quad, \qquad, \hfill etc. 
	%(or a blank line to force the subfigure onto a new line)
	\begin{subfigure}[b]{0.3\textwidth}
		\includegraphics[width=\textwidth]{figs/amplitude_modulation/dsb/carrier_signal_freq.png}
		\caption{Carrier signal}
		\label{fig:dsb_carrier_signal_freq}
	\end{subfigure}
	~ %add desired spacing between images, e. g. ~, \quad, \qquad, \hfill etc. 
	%(or a blank line to force the subfigure onto a new line)
	\begin{subfigure}[b]{0.3\textwidth}
		\includegraphics[width=\textwidth]{figs/amplitude_modulation/dsb/modulated_signal_freq.png}
		\caption{Modulated signal}
		\label{fig:dsb_modulated_signal_freq}
	\end{subfigure}
	\caption{Modulating a message signal onto a carrier signal with DSB modulation. Plots (a), (b) and (c) display signals in the time domain, whereas (d), (e) and (c) display the signals in the frequency domain. Note how the carrier signal frequency gets split into two smaller peaks. These are referred to as the upper and lower sidebands.}\label{fig:dsb_modulation}
\end{figure}

\subsubsection{Ordinary Amplitude Modulation}

\subsubsection{Single-Sideband Modulation}

\subsubsection{Vestigial-Sideband Modulation}


%\end{multicols}
\end{document}
