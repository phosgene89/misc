\documentclass[]{article}
\usepackage[parfill]{parskip}
\usepackage{amsmath}

%opening
\title{Notes for Mathematics for Physics: A Guided Tour for Graduate Students}
\author{Gregory Feldmann}

\begin{document}
\maketitle

\tableofcontents
\newpage

\section{Calculus of variations}

A functional $J$ is a map $J:C^{\infty}(\mathcal{R}) \rightarrow \mathcal{R}$. We restrict ourselves to functionals of the form

\begin{equation} J[y] = \int_{x_{1}}^{x_{2}} F(x,y,y^{(1)},y^{(2)},...)dx \label{functional_def} \end{equation}
where $y^{(n)}$ denotes the $n^{th}$ derivative of $y$ with respect to $x$.

\subsection{Minimising functionals of $y$ and $y_{x}$}

For $J=\int Fdx$, where $f$ depends only on $x$, $y$ and $y'$, we show how to derive the Euler-Lagrange equation and find the $y$ that optimises $J$. 

Let $\epsilon$ be a small, arbitrary real number, $\eta$ an arbitrary function of x and $y$ an arbitrary function of $x$ in $C^{\infty}$. A small perturbation in $y$ is given by $y = y^{*} + \epsilon \eta$, where $y^{*}$ is the unperturbed $y$. Also let $\delta J = J[y^{*} + \epsilon \eta] - J[y^{*}] $ and $\delta  F = \{ F(x, y, y^{1})-F(x,y^{*},y^{*(1)}) \}$, where $\delta$ is the variation operator. The change in $J$ associated with going from $y$ to $y^{*} + \epsilon \eta$ is given by
\begin{equation} \delta J = \int_{x_{1}}^{x_{2}} \delta  F dx \label{variational_functional_example}\end{equation}

A necessary condition for the minimisation of $J$ is $\delta J= 0$. We then use a Taylor expansion of $\delta F$ around $\epsilon = 0$, discarding all terms second order and above.

\begin{equation} \delta F \approx \epsilon \bigg(\frac{\partial F}{\partial y} \eta+ \frac{\partial F}{\partial y^{(1)}}\eta^{(1)}\bigg) \label{taylor_expansion_example} \end{equation}
Using integration by parts, we can find an alternate expression for $\frac{\partial F}{\partial y^{(1)}}\eta^{(1)}$

\begin{equation*} \int_{x_{1}}^{x_{2}} \frac{\partial F}{\partial y^{(1)}}\eta^{(1)}dx = \frac{\partial F}{\partial y^{(1)}}\eta \bigg|_{x_1}^{x_{2}} - \int_{x_{1}}^{x_{2}} \frac{d}{dx}\frac{\partial F}{\partial y^{(1)}}\eta dx\end{equation*}

As we assume $\epsilon =0$ at $x_{1}$ and $x_{2}$, $\frac{\partial F}{\partial y^{(1)}}\eta \bigg|_{x_1}^{x_{2}} = 0$. So we have

\begin{equation*} \frac{\partial F}{\partial y^{(1)}}\eta^{(1)} = - \frac{d}{dx}\frac{\partial F}{\partial y^{(1)}}\eta \end{equation*}

We substitute this into ($\ref{taylor_expansion_example}$) to obtain

\begin{equation*} \delta F \approx \epsilon \eta \bigg(\frac{\partial F}{\partial y} - \frac{d}{dx}\frac{\partial F}{\partial y^{(1)}}\bigg) \label{simplified_taylor_expansion} \end{equation*}

Substituting this expression into ($\ref{variational_functional_example}$) results in

\begin{equation*} \int_{x_{1}}^{x_{2}} \delta F dx = \int_{x_{1}}^{x_{2}} \epsilon \eta \bigg(\frac{\partial F}{\partial y} - \frac{d}{dx}\frac{\partial F}{\partial y^{(1)}}\bigg) dx = 0 \label{variational_eta_final} \end{equation*}

Supposing that $\eta$ and $\epsilon$ are not identically zero, we conclude that 

\begin{equation} \frac{\partial F}{\partial y} - \frac{d}{dx}\frac{\partial F}{\partial y^{(1)}} = 0 \label{euler_lagrange_eq1} \end{equation}

($\ref{euler_lagrange_eq1}$) is known as the \textit{Euler-Lagrange equation}. The expression on the left hand side is referred to as the functional derivative of $\delta J$ with respect to $y$.

\subsection{Functionals of higher order derivatives}
If $F$ is a function of higher order derivatives of $y$, e.g. $y^{(5)}$ or $y^{(26)}$, then the Euler-Lagrange equation is extended as follows. Suppose $F$ is a function of the $n^{th}$ order derivative of $y$. The Euler-Lagrange equation is

\begin{equation} \frac{\partial F}{\partial y} + \sum_{i=1}^{n} (-1)^{i}\frac{d^{i}}{dx^{i}} \bigg(\frac{\partial F}{\partial y^{(i)}}\bigg) = 0 \end{equation}  

\subsection{Minimising functionals of multiple functions}
When $F$ is a function of multiple functions $y_{i}$ and their derivatives $y_{ix}$, where each $y_{i}$ is a function of $x$, then we get a separate Euler-Lagrange equation for each $y_{i}$

\begin{equation} \frac{\partial F}{\partial y_{i}} - \frac{d}{dx}\frac{\partial F}{\partial y_{i}^{(1)}} = 0 \end{equation}

\subsection{Functionals of multiple functions and higher order derivatives}
Combining the previous two sections, the Euler-Lagrange equations for functionals of multiple functions $y_{i}$ and higher order derivatives $y_{i}^{(n)}$ are given by

\begin{equation} \frac{\partial F}{\partial y_{i}} + \sum_{k=1}^{n} (-1)^{k}\frac{d^{k}}{dx^{k}} \bigg(\frac{\partial F}{\partial y^{(k)}_{i}}\bigg) = 0 \end{equation}  

for each $i$ corresponding to a $y_{i}$.

\subsection{Examples}
test
\subsection{Variable end points}
test

\bibliographystyle{amsplain}

\bibliography{refs}

\section{Function spaces}

\section{Linear ordinary differential equations}

\section{Linear differential operators}

\section{Green functions}

\end{document}
