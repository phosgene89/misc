\documentclass[]{article}
\usepackage{amsmath}

%opening
\title{Reinforcement Learning Notes}
\author{Gregory Feldmann}

\begin{document}

\maketitle

\begin{abstract}

\end{abstract}

\section{Overview}
The essential elements of reinforcement learning are \textit{agents}, \textit{environments}, \textit{policies}, \textit{reward signals}, \textit{value functions} and \textit{models} of the environment. \textit{Agents} are entities that respond to the environment by performing actions. The \textit{environment} provides information to the agent. \textit{Policies} determine which actions an agent takes in response to the environment. A \textit{reward signal} determines how beneficial actions performed by the agent are (ie the reward to give an agent after it performs an action). The \textit{value function} determines how rewards accumulate long term. This differs from the reward signal in that reward signals are given in response to a single action, while value functions are a way of aggregating multiple rewards. Value functions are important because they allow agents to make considerations about the long term value of an action, rather than just taking into account the immediate reward. \textit{Models} are representations of the environment that agents can use for planning. \textit{Model free} methods in reinforcement learning exist, so a model is not necesarry.

\end{document}
